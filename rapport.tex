\documentclass[a4paper, 11pt, oneside]{article}

\usepackage[utf8]{inputenc}
\usepackage[T1]{fontenc}
\usepackage[french]{babel}
\AddThinSpaceBeforeFootnotes 
\FrenchFootnotes 
\usepackage[scaled=.90]{helvet}
\usepackage{mathptmx}
\usepackage{array}
\usepackage{shortvrb}
\usepackage{listings}
\usepackage[fleqn]{amsmath}
\usepackage{amsfonts}
\usepackage{fullpage}
\usepackage{enumerate}
\usepackage{graphicx}             % import, scale, and rotate graphics
\usepackage{subfigure}            % group figures
\usepackage{alltt}
\usepackage[hidelinks]{hyperref}
\hypersetup{
colorlinks=true,
linkcolor=blue,
filecolor=magenta,
urlcolor=cyan,
pdfpagemode=FullScreen,
}
\usepackage{url}
\usepackage{indentfirst}
\usepackage{eurosym}
\usepackage{listings}
\usepackage{color}
\usepackage[table,xcdraw,dvipsnames]{xcolor}

% Change le nom par défaut des listing
\renewcommand{\lstlistingname}{Codingstyle}

% Change la police des titres pour convenir à votre seul lecteur
\usepackage{sectsty}
\allsectionsfont{\sffamily\mdseries\upshape}
% Idem pour la table des matière.
\usepackage[nottoc,notlof,notlot]{tocbibind}
\usepackage[titles,subfigure]{tocloft}
\renewcommand{\cftsecfont}{\rmfamily\mdseries\upshape}
\renewcommand{\cftsecpagefont}{\rmfamily\mdseries\upshape}

\definecolor{mygray}{rgb}{0.5,0.5,0.5}
\newcommand{\coms}[1]{\textcolor{MidnightBlue}{#1}}

\lstset{
language=C, % Utilisation du langage C
commentstyle={\color{MidnightBlue}}, % Couleur des commentaires
frame=single, % Entoure le code d'un joli cadre
rulecolor=\color{black}, % Couleur de la ligne qui forme le cadre
stringstyle=\color{RawSienna}, % Couleur des chaines de caractères
numbers=left, % Ajoute une numérotation des lignes à gauche
numbersep=5pt, % Distance entre les numérots de lignes et le code
numberstyle=\tiny\color{mygray}, % Couleur des numéros de lignes
basicstyle=\tt\footnotesize,
tabsize=3, % Largeur des tabulations par défaut
keywordstyle=\tt\bf\footnotesize\color{Sepia}, % Style des mots-clés
extendedchars=true,
captionpos=b, % sets the caption-position to bottom
texcl=true, % Commentaires sur une ligne interprétés en Latex
showstringspaces=false, % Ne montre pas les espace dans les chaines de caractères
escapeinside={(>}{<)}, % Permet de mettre du latex entre des <( et )>.
inputencoding=utf8,
literate=
{á}{{\'a}}1 {é}{{\'e}}1 {í}{{\'i}}1 {ó}{{\'o}}1 {ú}{{\'u}}1
{Á}{{\'A}}1 {É}{{\'E}}1 {Í}{{\'I}}1 {Ó}{{\'O}}1 {Ú}{{\'U}}1
{à}{{\`a}}1 {è}{{\`e}}1 {ì}{{\`i}}1 {ò}{{\`o}}1 {ù}{{\`u}}1
{À}{{\`A}}1 {È}{{\`E}}1 {Ì}{{\`I}}1 {Ò}{{\`O}}1 {Ù}{{\`U}}1
{ä}{{\"a}}1 {ë}{{\"e}}1 {ï}{{\"i}}1 {ö}{{\"o}}1 {ü}{{\"u}}1
{Ä}{{\"A}}1 {Ë}{{\"E}}1 {Ï}{{\"I}}1 {Ö}{{\"O}}1 {Ü}{{\"U}}1
{â}{{\^a}}1 {ê}{{\^e}}1 {î}{{\^i}}1 {ô}{{\^o}}1 {û}{{\^u}}1
{Â}{{\^A}}1 {Ê}{{\^E}}1 {Î}{{\^I}}1 {Ô}{{\^O}}1 {Û}{{\^U}}1
{œ}{{\oe}}1 {Œ}{{\OE}}1 {æ}{{\ae}}1 {Æ}{{\AE}}1 {ß}{{\ss}}1
{ű}{{\H{u}}}1 {Ű}{{\H{U}}}1 {ő}{{\H{o}}}1 {Ő}{{\H{O}}}1
{ç}{{\c c}}1 {Ç}{{\c C}}1 {ø}{{\o}}1 {å}{{\r a}}1 {Å}{{\r A}}1
{€}{{\euro}}1 {£}{{\pounds}}1 {«}{{\guillemotleft}}1
{»}{{\guillemotright}}1 {ñ}{{\~n}}1 {Ñ}{{\~N}}1 {¿}{{?`}}1
}
\newcommand{\tablemat}{~}

%%%%%%%%%%%%%%%%% TITRE %%%%%%%%%%%%%%%%
% Complétez et décommentez les définitions de macros suivantes :
\newcommand{\intitule}{Complément de programmation \\ Projet 2: TAD \& Récursivité.}
\newcommand{\GrNbr}{34}
\newcommand{\PrenomUN}{Timothy}
\newcommand{\NomUN}{Smeers}
\newcommand{\PrenomDEUX}{Soline}
\newcommand{\NomDEUX}{Lèbre}
% Décommentez ceci si vous voulez une table des matières :
\renewcommand{\tablemat}{\tableofcontents}

%%%%%%%% ZONE PROTÉGÉE : MODIFIEZ UNE DES DIX PROCHAINES %%%%%%%%
%%%%%%%%            LIGNES POUR PERDRE 2 PTS.            %%%%%%%%
\title{INFO0947: \intitule}
\author{Groupe \GrNbr : \PrenomUN~\textsc{\NomUN}, \PrenomDEUX~\textsc{\NomDEUX}}
\date{\today}
\begin{document}
\maketitle
\newpage
\tablemat
\newpage
%%%%%%%%%%%%%%%%%%%% FIN DE LA ZONE PROTÉGÉE %%%%%%%%%%%%%%%%%%%%

\section{Définition de TAD}

	\subsection{Course\_t}
	
	\subsection{Escale\_t}
		\begin{tabular}{|p{17cm}|c}
			\hline
			\\
			
			\textbf{Types:}
				\begin{itemize}
					\item[] Escale
				\end{itemize}
			
			\textbf{Utilise:}
				\begin{itemize}
					\item[] float
					\item[] String
				\end{itemize}
			
			\textbf{Opérations\footnotemark{}:}
				\begin{itemize}
					\item[] \textcolor{red}{create\_stopover}:\textcolor{green}{float} $\times$ \textcolor{green}{float} $\times$ \textcolor{green}{char} $\rightharpoonup$ \textcolor{blue}{Escale}
					\item[] \textcolor{red}{calculate\_range}: \textcolor{green}{Escale} $\times$ \textcolor{green}{Escale} $\rightharpoonup$ \textcolor{blue}{float}
					\item[] \textcolor{red}{log\_time}: \textcolor{green}{Escale} $\times$ \textcolor{green}{float} $\rightharpoonup$ \textcolor{blue}{Escale}
					\item[] \textcolor{red}{free\_stopover}: $\rightharpoonup$ \textcolor{blue}{Escale}
					\item[] \textcolor{magenta}{get\_x}: \textcolor{green}{Escale} $\rightharpoonup$ \textcolor{blue}{float}
					\item[] \textcolor{magenta}{get\_y}: \textcolor{green}{Escale} $\rightharpoonup$ \textcolor{blue}{float}
					\item[] \textcolor{magenta}{get\_name}: \textcolor{green}{Escale} $\rightharpoonup$ \textcolor{blue}{String}
					\item[] \textcolor{magenta}{get\_time}: \textcolor{green}{Escale} $\rightharpoonup$ \textcolor{blue}{float}
				\end{itemize}
			
			\textbf{Préconditions:}
				\begin{itemize}
					\item[$\bullet$] $\forall \hspace{1mm} x \in float, \hspace{1mm} x \in [-90,90] \hspace{2mm} \&\& \hspace{2mm} \forall \hspace{1mm} y \in float, \hspace{1mm} y \in [-180,180]  \hspace{2mm} \&\&$ \newline $\forall \hspace{1mm} name \neq \oslash \Rightarrow create\_stopover(x, \hspace{1mm} y, \hspace{1mm} name)$
					\item[$\bullet$] $\forall \hspace{1mm} stopover, \hspace{1mm} secondStopover \in Escale \neq \oslash \Rightarrow calculate\_range(stopover, \hspace{1mm} secondStopover)$
					\item[$\bullet$] $\forall \hspace{1mm} stopover \in Escale \neq \oslash \hspace{2mm} \&\& \hspace{2mm} \forall \hspace{1mm} time \geq 0 \Rightarrow log\_time(stopover, \hspace{1mm} time) $
					\item[$\bullet$] $\forall \hspace{1mm} stopover \in Escale \neq \oslash \Rightarrow free\_stopover(stopover)$ 
					\item[$\bullet$] $\forall \hspace{1mm} stopover \in Escale \neq \oslash \Rightarrow get\_x(stopover)$ 
					\item[$\bullet$] $\forall \hspace{1mm} stopover \in Escale \neq \oslash \Rightarrow get\_y(stopover)$ 
					\item[$\bullet$] $\forall \hspace{1mm} stopover \in Escale \neq \oslash \Rightarrow get\_name(stopover)$ 
					\item[$\bullet$] $\forall \hspace{1mm} stopover \in Escale \neq \oslash \Rightarrow get\_time(stopover)$ 
				\end{itemize}
				
			\\	
			\hline
		\end{tabular}
		 
		\footnotetext{\textcolor{red}{Nom des opérations interne}} 
		\footnotetext{\textcolor{green}{Arguments}} 
		\footnotetext{\textcolor{blue}{Types de retour}} 
		\footnotetext{\textcolor{magenta}{Nom des opérations d'observation}} 
	
\end{document}